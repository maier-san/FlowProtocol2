\documentclass[12pt ,a4paper]{article}
\usepackage[ansi]{umlaute}
\usepackage[german]{babel}
\usepackage{fancyheadings}
\usepackage{geometry}
\usepackage[T1]{fontenc}
\usepackage{mathptm}
\usepackage{graphicx}
\title{Softwareentwicklung mit FlowProtocol 2}
\author{Wolfgang Maier}
%%
%% Layout
%%
\setlength{\parindent}{0mm}
\setlength{\parskip}{0.5ex plus0.2ex minus0.1ex}
\geometry{lmargin=35mm,rmargin=25mm,tmargin=25mm,bmargin=25mm}
\addtolength{\headheight}{2.5pt}
\pagestyle{fancy}
\renewcommand{\sectionmark}[1]{\markboth{\thesection.~#1}{}}
%\setlength{\headrulewidth}{0pt}
\rhead{\thepage}
\cfoot{}
%%
%% Commands
%%
\newcommand{\FPZ}{\textit{FlowProtocol 2}\ }
\newcommand{\Syntax}{\subsubsection*{Syntax}}
\newcounter{bspcount}
\newcommand{\Beispiel}{\stepcounter{bspcount}\subsubsection*{Beispiel~\arabic{bspcount}}}
\newcommand{\webbox}[1]{\par\setlength{\fboxsep}{3mm}\fbox{\begin{minipage}{12cm}\textsf{#1}\end{minipage}\par}\par\vspace{3mm}}
\newcommand{\webhead}[2]{{\large  #1}\\[2mm]#2\\[1mm]\rule{\linewidth}{1pt}\\[3mm]}
\newcommand{\webheadR}[1]{{\large  #1}\\}
\newenvironment{circlist}%
{\begin{list}{$\bigcirc$}{%
\setlength{\labelsep}{5pt}
\setlength{\topsep}{5pt}
\setlength{\itemsep}{0pt}
\setlength{\parsep}{0pt}
\setlength{\labelwidth}{1cm}
}}{\end{list}}
\newcounter{lcount}
\newenvironment{numlist}%
{\begin{list}{\textrm{\arabic{lcount}}}{\usecounter{lcount}%
\setlength{\labelsep}{5pt}
\setlength{\topsep}{5pt}
\setlength{\itemsep}{0pt}
\setlength{\parsep}{0pt}
\setlength{\labelwidth}{1cm}
}}{\end{list}}
\newcommand{\ZeigeFehlerItem}[4]{\item\textbf{#1}\\#2\ \textsl{#3}\\\texttt{#4}}
\newcommand{\FehlerBeschreibungItem}[2]{\item[#1]\textsl{#2}\\}
\begin{document}
\maketitle
\begin{center}
	\includegraphics{FlowLogo}
\end{center}
\newpage
\tableofcontents
\newpage
\section*{Vorwort}
Die vorliegende Anleitung beschreibt M�glichkeiten, die Anwendung \FPZ in der professionellen Softwareentwicklung einzusetzen und damit die besonderen Herausforderungen zu meistern, die sich in diesem Bereich stellen. An vielen kleinen und auch auch gr��eren Beispielen wird gezeigt, wie man Skripte in \FPZ erstellt und welche Bandbreite an Hilfsmitteln sich damit anfertigen lassen. Man lernt, Skripte als Ablageform f�r Wissen und Konventionen innerhalb eines Entwicklungsteams zu sehen und auch die t�gliche Praxis darauf auszurichten. 

\FPZ ist die komplett �berarbeitete Nachfolgeversion von FlowProtocol, das im Winter 2021/22 entwickelt wurde. Es handelt sich um eine kleine Anwendung. die �ber einen Browser bedient wird, und auf der Skripte ausgef�hrt werden k�nnen. Die Skripte bestehen aus einfachen Textdateien, die in einem beliebigen Editor erstellt, und in einer lokalen Verzeichnisstruktur verwaltet werden, mit der eine organisatorische oder aufgabenbezogene Gliederung abgebildet werden kann.
Bei der Ausf�hrung eines Skriptes werden Informationen abgefragt und �ber die Anweisungen im Skript verarbeitet. Die Ausgabe erfolgt ausschlie�lich als Ergebnisdokument im Browser, aus dem dann z.B. Textpassagen �ber die Zwischenablage weiterverwendet werden k�nnen. Bei der Ausf�hrung werden keine Dateien erstellt und ge�ndert.

Der Anwendungsfall, f�r den FlowProtocol urspr�nglich entwickelt wurde, ist die Erstellung von Checklisten, die durch Interaktion mit dem Benutzer auf einen individuellen Fall zugeschnitten werden, und so beliebig ins Detail gehen k�nnen, ohne unn�tige Eintr�ge aufzulisten. Ein Beispiel, an dem man das gut verdeutlichen kann, ist die Erstellung einer Packliste f�r eine Urlaubsreise. Ohne Kenntnis �ber die Art des Urlaubs, seine L�nge, Jahreszeit und Bet�tigungsm�glichkeiten, wird man sich entweder auf sehr allgemeine Dinge wie Unterhosen und Socken beschr�nken m�ssen, oder man wird Eintr�ge auflisten, die nur unter bestimmten Bedingungen relevant sind, wie die Wanderschuhe oder die Taucherbrille. 
�ber \FPZ k�nnen solche Bedingungen direkt abgefragt werden, so dass die Wanderschuhe nur dann aufgelistet werden, wenn auch die M�glichkeit zum Wandern besteht und genutzt werden soll. Solche Abfragen lassen sich auch ineinander verschachteln, so dass die Frage nach bergigem Gel�nde z.B.\ nur dann gestellt wird, wenn Wandern allgemein schon als relevant erkl�rt wurde.

Der gr��te Vorteil dieser Interaktion, abgesehen vom ma�geschneiderten Ergebnis, liegt jedoch nicht im Komfort, mit dem man seine Liste zusammenstellen kann, sondern darin, dass die Fragen einfach formuliert, und Abh�ngigkeiten direkt ber�cksichtigt werden k�nnen. 
\end{document}