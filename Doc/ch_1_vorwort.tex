\chapter*{Vorwort}
Die vorliegende Anleitung beschreibt Möglichkeiten, die Anwendung \FPZ\ in der professionellen Softwareentwicklung einzusetzen und damit die besonderen Herausforderungen zu meistern, die sich in diesem Bereich stellen. An vielen kleinen und auch auch größeren Beispielen wird gezeigt, wie man Skripte in \FPZ\ erstellt und welche Bandbreite an Hilfsmitteln sich damit bereitstellen lassen. Man lernt, Skripte als Ablageform für Wissen und Konventionen innerhalb eines Entwicklungsteams zu sehen und die eigenen Fähigkeiten in dieser Form weiterzugeben.

\FPZ\ ist die komplett überarbeitete Nachfolgeversion von FlowProtocol, das im Winter 2021/22 entwickelt wurde. Es handelt sich um eine kleine Anwendung. die über einen Browser bedient wird, und auf der Skripte ausgeführt werden können. Die Skripte bestehen aus einfachen Textdateien, die in einem beliebigen Editor erstellt werden können. Die Verwaltung der Skripte erfolgt in einer lokalen Verzeichnisstruktur, mit der eine organisatorische oder aufgabenbezogene Gliederung abgebildet werden kann.

Bei der Ausführung eines Skriptes werden Informationen über Eingabefelder abgefragt und über die Anweisungen im Skript verarbeitet. Die Ausgabe erfolgt ausschließlich als Ergebnisdokument im Browser, aus dem dann z.B.\ Textpassagen über die Zwischenablage weiterverwendet werden können. Alle eingegebenen Daten werden ausschließlich als Parameter in der URL verwaltet, es gibt keine angebundene Datenbank, keine Benutzerverwaltung und es werden durch die Anwendung keine Dateien erstellt und geändert.

Der Anwendungsfall, für den FlowProtocol ursprünglich entwickelt wurde, ist die Erstellung von Checklisten, die durch Interaktion mit dem Benutzer auf einen individuellen Fall zugeschnitten werden, und so beliebig ins Detail gehen können, ohne unnötige Einträge aufzulisten. Daran werden sich auch die ersten Beispiele in dieser Anleitung orientieren. Auf dieser Basis entstanden zahlreiche Skripte für den Product Owner, die für die verschiedenen Standardentwicklungen genau die benötigten Einstellungen und Informationen abfragten, die mit den Framework-Komponenten unseres Werkzeugkastens umgesetzt werden konnten, und die in ihrer Gesamtheit und mit ihrem Zusammenspiel das Designsystem unserer Software bildeten. Der Vorteil für den Product Owner bestand darin, dass ihm aufgrund der in den Skripten hinterlegten Abhängigkeiten immer nur die Optionen angeboten wurden, die für den jeweiligen Fall sinnvoll waren, und er so auch an alle Entscheidungen herangeführt wurde, die an der jeweiligen Stelle getroffen werden mussten. Das Ergebnisdokument bestand in diesem Fall aus einem sehr umfangreichen Userstory-Entwurf, in dem schon alle abgefragten Informationen und Entscheidungen eingearbeitet waren, und der mehr oder weniger nur noch um einige Benennungen und Aufzählungen angereichert werden musste. Schon bald wurde FlowProtocol um die Möglichkeit erweitert, auch Texteingaben abzufragen und diese in das Ergebnisdokument einzuarbeiten.

Die sehr systematisch und mit hohem Detailgrad aufgebauten Anforderungslisten in den Userstories konnten aufgrund der  sehr engen Anlehnung an die programmtechnische Umsetzung mit den Klassen und Funktionen aus dem Framework schon fast wie eine Entwicklungsanleitung gelesen werden, was lag also näher, als analoge Unterstützungswerkzeuge auch auf Entwicklungsseite zu schaffen, die den Programmierer dazu anleiteten, die richtigen Klassen zu verwenden und die am Ende idealerweise sogar fertigen Programmcode erstellten. Zu diesem Zweck wurden die Formatierungsmöglichkeiten so erweitert, dass man auch Codeschnipsel darstellen konnte, die sich einfach über die Zwischenablage in das aktuelle Projekt übernehmen ließen. Ebenso wurde der \fcmd{CamelCase}-Befehl geschaffen, mit dem Namen für Felder, Funktionen oder Variablen aus einer beliebigen Bezeichnung erzeugt werden können. Die Verwendung interaktiver Anleitungen mit Code-Generierung steigerte nicht nur die Effizienz bei der Umsetzung von Standardaufgaben, sie half auch bei der Sicherstellung von Einheitlichkeit und ist damit bis heute eine wichtige Säule in der Qualitätssicherung.

Auch wenn \FPZ\ selbst nicht direkt mit anderen Anwendungen interagiert, so kann in einem Skript praktisch jede URL mit beliebigen Parametern aufgebaut werden, so dass sowohl bei der Ausführung, als auch aus dem Ergebnisdokument der parametrisierte Aufruf anderer Web-, oder Intranet-Anwendungen möglich ist. Allen voran können Skripte auf diese Weise andere Skripte aufrufen, aber auch viele der im Entwicklungsumfeld eingesetzten Anwendungen für die Verwaltung von Vorgängen oder als Wiki bieten gute Steuerungsmöglichkeiten. Schon allein mit der Übergabe eines Suchbegriffs kommt man schon recht weit, und über einen MailTo-Link lassen sich sogar vollständig ausformulierte E-Mails vorbereiten. Die Codierung der Parameter für die URL lässt sich dabei sehr einfach mit dem \fcmd{UrlEncode}-Befehl umsetzen. Auch Desktop-Anwendungen lassen sich sehr gut in Anleitungen integrieren, insbesondere wenn sich damit ganze Verarbeitungsschritte allein mit geeigneten Kommandozeilenparametern ausführen lassen. Die Anleitung kann damit die kompletten Befehlszeilen vorbereiten, die man dann über die Zwischenablage ausführen kann.

Inzwischen umfasst der Befehlssatz von \FPZ\ alle notwendigen Befehle für den Aufbau von Programmen, wie If-Abfragen, For-Schleifen, sowie die Definition von Funktionen, die auch rekursiv aufgerufen werden können. Zusammen mit den verschiedenen Befehlen für das Rechnen mit Zahlen und Datum-Uhrzeit-Werten und zur Manipulation von Zeichenketten lassen sich mit sehr geringem Aufwand kleinere und größere nützliche Hilfsanwendungen schreiben, die unmittelbar auf jedem Arbeitsplatz verfügbar sind.

Der Nutzen von \FPZ\ innerhalb eines Teams oder einer Einrichtung erhöht sich in besonderem Maße dadurch, dass die Skripte von einer größeren Zahl an Kollegen gepflegt und erweitert werden, und dass der bestehenden Erfahrungsschatz auf diese Weise permanent durch neues Wissen erweitert wird. In einfacher Form lässt sich dies sogar hinbekommen, ohne dass dafür Programmierkenntnisse vorausgesetzt werden, indem ein Skript die für seine eigene Erweiterung notwendigen Informationen abfragt, und daraus den resultierenden Programmcode samt Einbauanleitung selbst erzeugt. Auch diese Methode wird an einem Beispiel beschrieben.

Bis dahin werden aber erst einmal die grundlegenden Befehle und Funktionsweisen beschrieben, um einen einfachen Einstieg in \FPZ\ zu ermöglichen. Ich freue mich über jeden, der mit Hilfe dieser kleinen Anleitung auf Entdeckungstour geht, und wünsche viel Spaß und einen hoffentlich nutzbringenden Einsatz.

Zuletzt möchte ich noch Danke sagen, an alle Kollegen bei easySoft, die \FPZ\ bei der täglichen Arbeit eingesetzt, und durch ihr Feedback zur permanenten Verbesserung beigetragen haben, insbesondere an \'Eric Louvard, der dort für eine permanent robuste und leicht aktualisierbare Installation der Anwendung gesorgt hat. Vielen Dank auch an alle Entwickler und Mitentwickler der zahlreichen Open-Source-Produkte, die ich bei der Entwicklung und der Erstellung dieser Dokumentation mit viel Freude genutzt habe.

\hspace*{\fill}Wolfgang Maier

\newpage

